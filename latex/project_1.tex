%% LyX 2.3.4.2 created this file.  For more info, see http://www.lyx.org/.
%% Do not edit unless you really know what you are doing.
\documentclass[11pt]{article}
\usepackage[utf8]{inputenc}
\usepackage[a4paper]{geometry}
\geometry{verbose}
\usepackage{fancyhdr}
\pagestyle{fancy}
\usepackage{float}
\usepackage{amsmath}

\makeatletter

%%%%%%%%%%%%%%%%%%%%%%%%%%%%%% LyX specific LaTeX commands.
\floatstyle{ruled}
\newfloat{algorithm}{tbp}{loa}
\providecommand{\algorithmname}{Algorithm}
\floatname{algorithm}{\protect\algorithmname}

%%%%%%%%%%%%%%%%%%%%%%%%%%%%%% User specified LaTeX commands.
% !TEX TS-program = pdflatex
% !TEX encoding = UTF-8 Unicode

% This is a simple template for a LaTeX document using the "article" class.
% See "book", "report", "letter" for other types of document.

% use larger type; default would be 10pt

% set input encoding (not needed with XeLaTeX)

%%% Examples of Article customizations
% These packages are optional, depending whether you want the features they provide.
% See the LaTeX Companion or other references for full information.

%%% PAGE DIMENSIONS
% to change the page dimensions
 % or letterpaper (US) or a5paper or....
% \geometry{margin=2in} % for example, change the margins to 2 inches all round
% \geometry{landscape} % set up the page for landscape
%   read geometry.pdf for detailed page layout information

\usepackage{graphicx}% support the \includegraphics command and options

% \usepackage[parfill]{parskip} % Activate to begin paragraphs with an empty line rather than an indent

%%% PACKAGES
\usepackage{subfig} % make it possible to include more than one captioned figure/table in a single float
\usepackage{booktabs}% for much better looking tables
\usepackage{array}% for better arrays (eg matrices) in maths
\usepackage{paralist}% very flexible & customisable lists (eg. enumerate/itemize, etc.)
\usepackage{verbatim}% adds environment for commenting out blocks of text & for better verbatim
% make it possible to include more than one captioned figure/table in a single float
% These packages are all incorporated in the memoir class to one degree or another...

%%% HEADERS & FOOTERS
\usepackage{fancyhdr}% This should be set AFTER setting up the page geometry
 % options: empty , plain , fancy
\renewcommand{\headrulewidth}{0pt} % customise the layout...
\lhead{}\chead{}\rhead{}
\lfoot{}\cfoot{\thepage}\rfoot{}

%%% SECTION TITLE APPEARANCE
\usepackage{sectsty}
\allsectionsfont{\sffamily\mdseries\upshape} % (See the fntguide.pdf for font help)
% (This matches ConTeXt defaults)

%%% ToC (table of contents) APPEARANCE
\usepackage[nottoc,notlof,notlot]{tocbibind}% Put the bibliography in the ToC
\usepackage[titles,subfigure]{tocloft}% Alter the style of the Table of Contents
\renewcommand{\cftsecfont}{\rmfamily\mdseries\upshape}
\renewcommand{\cftsecpagefont}{\rmfamily\mdseries\upshape} % No bold!

\usepackage{amsfonts}
%\usepackage{mathcal}

% Pseudocode
\usepackage{algorithm}
\usepackage{algpseudocode}[noend]

\usepackage{subfig}

%%% END Article customizations

%%% The "real" document content comes below...

\title{Project 1 in FYS3150}
\author{Simon Halstensen, Carl Fredrik Nordbø Knutsen, Jan Harald Aasen \& Didrik Sten Ingebrigtsen}
\date{05.09.2021} % Activate to display a given date or no date (if empty),
         % otherwise the current date is printed 

\makeatother

\begin{document}
\maketitle In this project we are solving the following equation:
\begin{equation}
-\dfrac{d^{2}u}{dx^{2}}=f(x)
\end{equation}
We also know that: 
\begin{itemize}
\item $f(x)=100e^{-10x}$ 
\item $x\in[0,1]$ 
\item $u(0)=u(1)=0$ 
\end{itemize}

\section*{Exercise 1}

I will check that 
\begin{equation}
u(x)=1-(1-e^{-10})x-e^{-10x}
\end{equation}
is a solution to (1) by differentiating $u(x)$ twice.

\[
\dfrac{d^{2}u}{dx^{2}}=\dfrac{d}{dx}(\dfrac{du}{dx})=\dfrac{d}{dx}(-(1-e^{-10})-(-10)e^{-10x})
\]
And since the derivative of a constant is 0, we get that: 
\[
\dfrac{d^{2}u}{dx^{2}}=\dfrac{d}{dx}(10e^{-10x})=-100e^{-10x}
\]
It immediately follows that 
\[
-\dfrac{d^{2}u}{dx^{2}}=100e^{-10x}
\]
This shows that (2) is a solution to equation (1).
This solution also satisfies the boundary conditions specified, as:
\[u(0) = 1 - (1- e^{-10})0 -e^{-10\cdot 0} = 1-1 = 0\]
and
\[u(1) = 1 - (1- e^{-10})1 -e^{-10\cdot 1} = 0\]

\section*{Exercise 2}

The program main.cpp evaluates the exact function \(u(x)\) from exercise 1, at points between 0 and 1. 
It writes the \(x\)-values and \(u(x)\)-values to a .csv-file, named exact\_evaluated.csv.
The python script read\_file\_and\_plot.py reads the values from the .csv-file, and plots the function (see figure 1).
\begin{figure}[htbp]
\centerline{\includegraphics{plots/exact_function_u(x).pdf}}
\caption{Plot of the exact function \(u(x) = 1 - (1- e^{-10})x -e^{-10x}\)}
\label{fig}
\end{figure}

\section*{Exercise 3}

I will derive a discretized version of equation (1) by finding a discretized
approximation of $\dfrac{d^{2}u}{dx^{2}}=u''(x)$. Let $h$ be a step
size, and let $a$ be a point such that $a\in[h,1-h]$. Firstly, evaluate
the 3rd degree Taylor expansion of u(x) about the point $a$ in the
points $a+h$ and $a-h$. 
\[
u(a+h)=u(a)+u'(a)\cdot h+\dfrac{1}{2}u''(a)\cdot h^{2}+\dfrac{1}{6}u'''(a)\cdot h^{3}+\mathcal{O}(h^{4})
\]
\[
u(a-h)=u(a)+u'(a)\cdot(-h)+\dfrac{1}{2}u''(a)\cdot h^{2}+\dfrac{1}{6}u'''(a)\cdot(-h)^{3}+\mathcal{O}(h^{4})
\]
Next, add the two equations, giving the following equality. 
\[
u(a+h)+u(a-h)=2u(a)+u''(a)\cdot h^{2}+\mathcal{O}(h^{4})
\]
The equation can be solved for $u''(a)$ 
\[
u''(a)=\dfrac{u(a+h)-2u(a)+u(a-h)}{h^{2}}+\mathcal{O}(h^{2})
\]
Assuming a sufficiently small value for h, we can approximate and
discretize with $u(ih)\approx v_{i}$. Here, $i\in\{0,1,..,n\}$ (meaning
$n=\dfrac{1}{h}$), and: 
\[
u''(ih)=\dfrac{v_{i+1}-2v_{i}+v_{i-1}}{h^{2}}
\]
Using equation (1), we can rewrite: 
\begin{equation}
h^{2}\cdot f(ih)=-v_{i+1}+2v_{i}-v_{i-1}
\end{equation}
Which is a discretized version of equation (1) with the following
conditions: 
\begin{itemize}
\item $v_{0}=u(0)=0$ 
\item $v_{n}=u(1)=0$. 
\end{itemize}

\section*{Exercise 4}

We will show that you can write the discretized equation as a matrix
equation:

\[
\boldsymbol{A}\vec{v}=\vec{g}
\]
We have eq. (3) from exercise 3, with $i=1,2,\dots,n$. 
\[
\begin{array}{cccccccc}
-v_{0} & 2v_{1} & -v_{2}\\
 & -v_{1} & 2v_{2} & -v_{3}\\
 &  & -v_{2} & 2v_{3} & -v_{4}\\
 &  & \dots & \dots & \dots & \dots & \dots\\
 &  &  &  & -v_{i-3} & 2v_{n-2} & -v_{n-1}\\
 &  &  &  &  & -v_{n-2} & 2v_{n-1} & -v_{n}
\end{array}=h{{}^2}\begin{array}{c}
f_{1}\\
f_{2}\\
f_{3}\\
\vdots\\
f_{n-2}\\
f_{n-1}
\end{array}
\]
we know $v_{0}$and $v_{n}$

\[
\begin{array}{cccccc}
2v_{1} & -v_{2}\\
-v_{1} & 2v_{1} & -v_{3}\\
 & -v_{2} & 2v_{1} & -v_{4}\\
 & \dots & \dots & \dots & \dots & \dots\\
 &  &  & -v_{n-3} & 2v_{n-2} & -v_{n-1}\\
 &  &  &  & -v_{n-2} & 2v_{n-1}
\end{array}=h{{}^2}\begin{array}{c}
f_{1}+v_{0}\\
f_{2}\\
f_{3}\\
\vdots\\
f_{n-2}\\
f_{n-1}+v_{n}
\end{array}\equiv\begin{array}{c}
g_{1}\\
g_{2}\\
g_{3}\\
\vdots\\
g_{n-2}\\
g_{n-1}
\end{array}
\]
We can then seperate out $\vec{v}$
\[
\left[\begin{array}{cccccc}
2 & -1\\
-1 & 2 & -1\\
 & -1 & 2 & -1\\
 & \dots & \dots & \dots & \dots & \dots\\
 &  &  & -1 & 2 & -1\\
 &  &  &  & -1 & 2
\end{array}\right]\left[\begin{array}{c}
v_{1}\\
v_{2}\\
v_{3}\\
\vdots\\
v_{n-2}\\
v_{n-1}
\end{array}\right]=\left[\begin{array}{c}
g_{1}\\
g_{2}\\
g_{3}\\
\vdots\\
g_{n-2}\\
g_{n-1}
\end{array}\right]
\]
We now have a known $A$ and $\bar{g}$, so we can solve for $\bar{v}$.

\section*{Exercise 5}

\subsection*{a)}

When the matrix equation in exercise 4 is solved, you get approximate solutions for all the inner points \(x_i\) on the interval \((0, 1)\). That is, you get solutions for \(v\) for all points except the first point and the last point. Therefore, if A is a \(n \times n\)-matrix, the complete solution \(\vec{v}^*\) must be of length \(m=n+2\), when you include \(v_0\) and \(v_{n+1}\) (note that we use a different notation in this problem, with n being the number of intervals and not the number of points).

\subsection*{b)}

The \(n\) equations give solutions \(v_i\) for all the inner points \(x_i\), as explained in exercise a). We do not solve for \(v_{n+1}\) or \(v_0\), as these values are already known (and necessary to compute values of \(v_i\) for \(i \in \{ 1, 2, ..., n\}\) with this method).

\section*{Exercise 6}

\subsection*{a)}

In this exercise, we want to formulate the algorithm for solving $Ax=g$
for a general tridiagonal $A$. This is done in {[}alg \ref{alg:general}{]}.

\begin{algorithm}
\caption{Algorithm for solving $Ax=g$ for a general tridiagonal matrix $A$.
$a$, $b$ and $c$ represent the sub-, main- and superdiagonal. Solving
it means taking in $A$ and $g$, and returning $x$.}
\begin{algorithmic}[0] \Procedure{tridiagonal solver}{a, b,
c, g, N} \State $\tilde{b}_{0}\gets b_{0}$ \State $\tilde{g}_{0}\gets g_{0}$
\For{$i\in(1,N)_{\mathbb{N}}$} \State $\tilde{b}_{i}\gets b_{i}-\frac{a_{i}}{\tilde{b}_{i-1}}c_{i-1}$
\State $\tilde{g}_{i}\gets g_{i}-\frac{a_{i}}{\tilde{b}_{i-1}}\tilde{g}_{i-1}$
\EndFor \State $x_{N}\gets\frac{\tilde{g}_{N}}{\tilde{b}_{N}}$
\For{$i\in(N-1,0)_{\mathbb{N}}$} \State $x_{i}\gets\frac{\tilde{g}_{i}-c_{i}x_{i+1}}{\tilde{b}_{i}}$
\EndFor \State \Return $x$ \EndProcedure \end{algorithmic} \label{alg:general} 
\end{algorithm}

\subsection*{b)}
The number of floating point operations (FLOPs) in the general algorithm in [alg \ref{alg:general}] is $2 \cdot 3 N = 6N$, where $N$ is the size of the matrix, for forward substitution. For back substitution, we have $3 N$ FLOPs. In total, the algorithm has $9N = \mathcal{O}(N)$ FLOPs. 

\section*{Exercise 7}

To be added

\section*{Exercise 8}

To be added

\section*{Exercise 9}

In this exercise, we want to specialize our algorithm from problem 6, to the case where our $A$ matrix is specified by the signature $(-1, 2, -1)$. This means that the matrix is tridiagonal, and with $a = (-1, -1, \ldots, -1)$, $b = (2, 2, \ldots, 2)$ and $c = (-1, -1, \ldots, -1)$.

\subsection*{a)}
Firstly, we want to describe how our specialized algorithm looks. If we start of with our general algorithm {[}alg \ref{alg:general}{]}, and set $a$, $b$, and $c$ to be our specific vectors, we find that $\tilde{b}$ is
\begin{align*}
\tilde{b}_i & = b_i - \frac{a_i}{\tilde{b}_{i-1}} c_{i-1} = 2 - \frac{(-1)}{\tilde{b}_{i-1}} \cdot (-1) \\
& = 2 - \frac{1}{\tilde{b}_{i-1}} = \begin{cases} \frac{i+3}{i+2} & i > 1 \\ 2 & i = 1 \end{cases}
\end{align*}

In the expression for $v$, we also retrieve elements from the $c$ vector, which now always gives the value $-1$. Therefore, it can be simplified slightly:
\begin{align*}
v_i & = \frac{\tilde{g}_i - c_i v_{i+1}}{\tilde{b}_i} = \frac{\tilde{g}_i + v_{i+1}}{\tilde{b}_i}
\end{align*}

$\tilde{g}$ can also be rewritten, and $v$ can be worked more on, so that neither retrieve data from $\tilde{b}$, making the vector obsolete. This will store and retrieve less data, but require more FLOPs, so we choose not to.

\subsection*{b)}
This new algorithm, which calculates $\tilde{b}$ in a simpler way, saves $2N$ FLOPs through simpler calculation of $\tilde{b}$, and $N$ for $v$, meaning the full algorithm goes from $9N$ to $6N$ FLOPs. This is still $\mathcal{O}(N)$, so the improvement is likely noticeable, but not very important.

% Likewise, by inserting our new $\tilde{b}$, $\tilde{g}$ becomes

% \begin{align*}
% \tilde{g}_i & = g_i + \frac{\tilde{g}_{i-1}}{\tilde{b}_{i-1}} = \begin{cases} g_i + \frac{i+1}{i+2} \tilde{g}_{i-1} & i > 2 \\ g_i + \frac{\tilde{g}_{i-1}}{2} & i = 2 \\ g_i & i = 1 \end{cases} \\
% \end{align*}

\section*{Exercise 10}

To be added

\section*{Exercise 11}

To be added

The number of floating point operations (FLOPs) in the general algorithm
in {[}alg \ref{alg:general}{]} is $2\cdot3N=6N$, where $N$ is the
size of the matrix, for forward substitution. For back substitution,
we have $3N$ FLOPs. In total, the algorithm has $9N=\mathcal{O}(N)$
FLOPs.
\end{document}
